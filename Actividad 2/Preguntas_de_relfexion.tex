\documentclass{article}
\usepackage[spanish]{babel}
\usepackage[utf8]{inputenc}
\usepackage{graphicx}
\usepackage{setspace}
\usepackage[usenames]{color}

\begin{document}
\title{Preguntas de reflexión}
\author{Campos Quiñonez Jorge Andres}
\maketitle

\begin{enumerate}
\item \textbf{¿Cual es tu primera impresión del uso de bash/Emacs?}\\
Tengo preferencia hacia una manera gráfica de hacer las cosas en lugar de la utilización de comandos.

\item \textbf{¿Ya lo habías utilizado?}\\
En la materia de Lenguaje y Programación en Fortran.

\item \textbf{¿Qué cosas se te dificultaron más en bash/Emacs?  }\\
Combinar ambos en uno solo, antes los utilizada a cada uno por su cuenta.

\item \textbf{¿Qué ventajas les ves a Emacs?  }\\
Hasta ahora, ninguna. Anteriormente utilizaba gedit, ambos pueden hacer lo mismo.

\item \textbf{¿Qué es lo que mas te llamó la atención en el desarrollo de esta actividad?}\\
La creación de un ejecutable que accedía a la base de datos de la Universidad de Wyoming y descargaba los datos.

\item \textbf{¿Qué cambiarías en esta actividad?}\\
Nada

\item \textbf{¿Que consideras que falta en esta actividad? }\\
Nada

\item \textbf{¿Puedes compartir alguna referencia nueva que consideras util y no se haya contemplado? }\\
Aún sigo aprendiendo a usar LaTex, sigue siendo consumidor de tiempo

\item \textbf{¿Algún comentario adicional que desees compartir? }\\
5mentarios.
\end{enumerate}





\end{document}