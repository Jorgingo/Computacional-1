\documentclass{article}
\usepackage[spanish]{babel}
\usepackage{natbib}
\usepackage{url}
\usepackage[utf8x]{inputenc}
\usepackage{amsmath}
\usepackage{float}
\usepackage{subfig}
\usepackage{graphicx}
\graphicspath{{images/}}
\usepackage{fancyhdr}
\usepackage{vmargin}
\usepackage{mathtools}
\usepackage{amssymb} 
\usepackage{enumitem}
\usepackage{makeidx}
\usepackage{hyperref}
\usepackage[none]{hyphenat}
\usepackage{setspace}
\title{Teoría del Caos y el Mapeo Logístico}	


\makeatletter  
\let\thetitle\@title
\let\theauthor\@author
\let\thedate\@date										
\makeatother

\pagestyle{fancy}
\fancyhf{} %%
\lhead{\thetitle}
\cfoot{\thepage}
\usepackage{setspace}
\begin{document}
%%%%%%%%%%%%%%%%%%%%%%%%%%%%%%%%%%%%%%%%%%%%%%%%%%%%%%%%%%%%%%%%%%%%%%%%%%%%%%%%%%%%%%%%%
\begin{titlepage}
\centering
  \vspace*{0.5 cm}
   \includegraphics[scale = 0.4]{logo.png}\\[0.5 cm]% University Logo
    \textsc{\LARGE Universidad de Sonora}\\[1.0 cm]	% University Name
	\textsc{\LARGE División de Ciencias Exactas y Naturales}\\[0.5 cm]	
    
	\textsc{\LARGE Física computacional}\\
    \textsc{\Large Carlos Lizarraga Celaya}\\ [0.5 cm]
    \rule{\linewidth}{0.2 mm} \\[0.4 cm]
	{ \huge \bfseries \thetitle}\\
	\rule{\linewidth}{0.2 mm} \\[0.5 cm]
    \textsc{\Large Campos Quiñonez Jorge Andres} \\[0.25 cm]
   \textsc {\large 17 de Mayo del 2017} 	

	
 
	\vfill
	
\end{titlepage}
\pagebreak

\newpage

%%%%%%%%%%%%%%%%%%%%%%%%%%%%%%%%%%%%%%%%%%%%%%%%%%%%%%%%%%%%%%%%%%%%%%%%%%%%%%%%%%%%%%%%%
%%%%%%%%%%%%%%%%%%%%%%%%%%%%%%%%%%%%%%%%%%%%%%%%%%%%%%%%%%%%%%
\pagebreak

\pagebreak
\onehalfspacing

\section*{\Large Modelo logístico}
En esta actividad, revisamos el artículo publicado por Geoff Boeing donde el realiza diversos mapeos logísticos donde compara el comportamiento de un sistema caótico, con el comportamiento de un sistema aleatorio y así mostrar como el sistema caótico se encuentra verdaderamente determinado y posee cierto comportamiento en base a sus condiciones iniciales. Lo que nosotros hicimos, fue replicar las gráficas utilizadas en ese artículo y de manera adicional, también se replicaron varios ejemplos de gráficos tridimensionales.\\

En este reporte incluiremos las gráficas obtenidas que provienen del artículo y los ejemplos de las gráficas tridimensionales, se encuentran en la carpeta 'Actividad 9' en github.

\begin{figure}[h]
\centering
\includegraphics[scale=0.7]{chaos-vs-random-line.png}
\end{figure}

\begin{figure}[h]
\centering
\includegraphics[scale=0.7]{logistic-map-bifurcation-0.png}
\end{figure}

\begin{figure}[h]
\centering
\includegraphics[scale=0.7]{Mapa_log_stico_de_un_atractor.png}
\end{figure}

\begin{figure}[h]
\centering
\includegraphics[scale=0.7]{3d-logistic-map-attractor-1}
\end{figure}

\begin{figure}[h]
\centering
\includegraphics[scale=0.7]{logistic-attractor-chaos-random.png}
\end{figure}

\begin{figure}[h]
\centering
\includegraphics[scale=0.7]{logistic-attractor-chaos-random-3d}
\end{figure}


\end{document}
