\documentclass{article}
\usepackage[spanish]{babel}
\usepackage[utf8]{inputenc}
\usepackage{graphicx}
\usepackage{setspace}
\usepackage[usenames]{color}

\begin{document}
\begin{doublespace}
\title{\includegraphics[scale=0.40]{escudo-gde-trans.png}\\ \Huge La estructura de la atmosfera\\Universidad de Sonora\\Física Computacional 1}
\author{\huge Campos Quiñonez Jorge Andres}
\date{\Large 29 de Enero de 2017}
\maketitle
\end{doublespace}

\newpage
\mbox{}
\thispagestyle{empty}
\newpage

\section{\huge Introducción}
\Large La atmosfera, algo que recubre al planeta tierra y que permite la vida en el mismo, nos brinda protección de los rayos UV además de proveernos el oxigeno necesario para sustentar la vida animal y vegetal. Compuesto de varias capas, todas con propiedades que los diferencian entre ellos. En este texto hablaremos sobre las distintas capas de posee la atmosfera de la tierra, todas que se encuentran en distintas alturas desde la superficie de la tierra, el como una es cada vez menos densa que la anterior, las condiciones que se presentan en cada una de ellas, la manera en que estas interactuan entre ellas, los fenomenos físicos que ocurren en ellas, que tipos de variables se utilizan para realizar las mediciones y con que instrumentos se realizan estas mediciones. 

\newpage

\section{\Huge Desarrollo}
\hfill
\begin{flushleft}
\title{\huge La atmosfera}
\end{flushleft}


Para que la tierra sea capaz de sustentar la vida que ahora o que en un futuro va a poseer, está necesita cumplir ciertos requerimientos. Uno de ellos es brindar a los seres vivos una capa protectora de gases que poseen la habilidad de sustentar vida además y muchas otras cosas. Aquí es donde entra en papel la atmosfera.
Una atmosfera es una capa de gases que recubren la corteza de cualquier planeta brindandole determinadas caracteristicas. Dependiendo de la composición de estos gases en la atmosfera,estos hacen posible la exitencia de minerales, vegetales o animales además de cumplir con la función de proteger de los rayos solares. Estos gases se ven atraidos al cuerpo celeste debido a la fuerza de gravedad.\\
\hfill 
%Sangría dejo de funcionar
\\Existen algunos planetas que parecen estar formados puramente de gases, eso indica que su atmosfera es increiblemente profunda. 
En el caso del planeta tierra, nuestra atmosfera posee una altura de alrededor de 10,000km, pero la mayor concentración de gases de la atmosfera se encuentra a los 5.5km de altura debido a la fuerza de compresión de la gravedad. Antes de los 15km de altura se encuentra el 95\% de la materia atmosferica.\\
\hfill \\
\hfill \\
\hfill \\ %todo esto es para que no se vea cortada la información


La atmosfera terrestre está compuesta principalmente por los siguientes compuestos:
\begin{itemize}
\item \textcolor{blue}{\textbf {Nitrogeno:}} Constituyendo el 78\% del aire, este es el gas más común presente en la atmosfera.
\item \textcolor{blue}{\textbf {Oxígeno:}} Representa el 21\% de los gases totales en la atmosfera. Requerido por la mayoría de la vida animal para vivir.
\item \textcolor{blue}{\textbf {Dióxido de carbono:}} El gas que se encarga de contener a los rayos del sol y mantener temperaturas adecuadas para sustentar la vida. Es 0.03\% de la materia en el aire.
\item  \textcolor{blue}{\textbf {Varios gases:}} Representan el porcentaje faltante de gases que componen al aire en la atmosfera, con el Argón siendo el de mayor cantidad, con un total de 0.09\% del volumen del aire.
\end{itemize}
La atmosfera terreste, a su vez, está conformada por 5 capas, cada una de ellas depende de la altura en la que se encuentra y lo que contienen aparte de su función. Adelante se encuentran enlistadas en orden de altura, con la primera siendo la más cercana a la corteza terrestre.\\

\subsection{\huge Toposfera}
Es la sección de la atmosfera más cercana a la corteza de la misma, con una altura máxima de hasta 18km en el ecuador. Contiene hasta el 75\% de la masa gaseosa de la atmosfera. Esta es la zona de las nubes y fenónemos climáticos como las lluvias, vientos, cambios de temperatura, etc.\\

\subsection{\huge Estratosfera}
Esta sección de la atmosfera se encuentra entre los 10 y 50km. Aquí es donde los gases se concentran en capas dependiendo de su peso. La más popular es la capa de Ozono, que es la encargada de proteger a la tierra de los rayos Ultravioleta que provienen del Sol. Esta capa también actúa como la reguladora de temperatura de la tierra.\\

\subsection{\huge Mesosfera}
Es la capa de la atmosfera que se encuentra entre los 50 y 80km de altura. Contiene únicamente el 0.1\% del volumen total de la atmosfera. Aquí es donde destacan lo que son las la ionización y las reacciones químicas que ocurren en ella.\\

\subsection{\huge Ionosfera}
La Ionosfera o Termosfera se extiende desde una altura de 80km hasta 640km o más. En esta existen subcapas formadas por átomos cargados eléctricamente, que se les llaman iones. Como esta es una capa conductora de electricidad, posibilita las transmisiones de radio y televisión por su propiedad de reflejar las ondas electromagnéticas. El gas que predomina en esta capa es el nitrógeno. Además de ser donde se produce la destrucción de los meteoritos que llegan a la Tierra.\\

\subsection{Exosfera}
Es la capa exterior de la atmosfera, que va desde los 600km hasta los 9,600km de altura y es donde el aire pierde sus cualidades. En esta misma se encuentra la magnetosfera, la cual es la que representa el campo magnetico que posee el planeta tierra.\\

Algo que debemos notar de todas estas capas, son las varaciones que se dan en presión, temperatura y la densidad que poseen dependiendo de la altura de la superficie terrestre. En la siguiente tabla se muestran determinadas variaciones.\\

\includegraphics[scale=1]{Tabla.PNG}

\newpage

\section{\Huge Conclusión}
Después de lo visto con anterioridad, podemos determinar que la atmosfera es una parte que posee una extrema importancia para todo cuerpo celeste. En el caso del planeta tierra, si no fuera por las condiciones perfectas en las que ahorita tenemos nuestra atmosfera después de millones de años de evolución que ha tenido, no seriamos capaces de estar aquí vivos. En años recientes se ha dado mucho descuido de nuestro planeta, eso incluye la atmosfera. Si no mantenemos un cuidado adecuado se pueden alterar las funciones de distintas capas; como lo hemos notado con el calentamiento global. De ahora en adelante, se espera un mayor cuidado en base a estos acontecimientos.


\end{document}
