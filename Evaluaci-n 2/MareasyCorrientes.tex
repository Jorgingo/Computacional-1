\documentclass{article}
\usepackage[spanish]{babel}
\usepackage{natbib}
\usepackage{url}
\usepackage[utf8x]{inputenc}
\usepackage{amsmath}
\usepackage{float}
\usepackage{subfig}
\usepackage{graphicx}
\graphicspath{{images/}}
\usepackage{fancyhdr}
\usepackage{vmargin}
\usepackage{mathtools}
\usepackage{amssymb} 
\usepackage{enumitem}
\usepackage{makeidx}
\usepackage{hyperref}
\usepackage[none]{hyphenat}
\usepackage{setspace}
\title{Evaluación 2}	


\makeatletter  
\let\thetitle\@title
\let\theauthor\@author
\let\thedate\@date										
\makeatother

\pagestyle{fancy}
\fancyhf{} %%
\lhead{\thetitle}
\cfoot{\thepage}
\usepackage{setspace}
\begin{document}
%%%%%%%%%%%%%%%%%%%%%%%%%%%%%%%%%%%%%%%%%%%%%%%%%%%%%%%%%%%%%%%%%%%%%%%%%%%%%%%%%%%%%%%%%
\begin{titlepage}
\centering
  \vspace*{0.5 cm}
   \includegraphics[scale = 0.4]{logo.png}\\[0.5 cm]% University Logo
    \textsc{\LARGE Universidad de Sonora}\\[1.0 cm]	% University Name
	\textsc{\LARGE División de Ciencias Exactas y Naturales}\\[0.5 cm]	
    
	\textsc{\LARGE Física computacional}\\
    \textsc{\Large Carlos Lizarraga Celaya}\\ [0.5 cm]
    \rule{\linewidth}{0.2 mm} \\[0.4 cm]
	{ \huge \bfseries \thetitle}\\
	\rule{\linewidth}{0.2 mm} \\[0.5 cm]
    \textsc{\Large Campos Quiñonez Jorge Andres} \\[0.25 cm]
   \textsc {\large 26 de Abril del 2017} 	

	
 
	\vfill
	
\end{titlepage}
\pagebreak

\newpage

%%%%%%%%%%%%%%%%%%%%%%%%%%%%%%%%%%%%%%%%%%%%%%%%%%%%%%%%%%%%%%%%%%%%%%%%%%%%%%%%%%%%%%%%%
%%%%%%%%%%%%%%%%%%%%%%%%%%%%%%%%%%%%%%%%%%%%%%%%%%%%%%%%%%%%%%
\pagebreak

\pagebreak
\onehalfspacing
\begin{enumerate}
\item \textbf{De los datos proporcionados, utiliza una transformada discreta de Fourier, para encontrar la frecuencia del ciclo principal. Muestra una gráfica con los principales modos encontrados.}
\begin{figure}[h]
\centering
\includegraphics[scale=1]{Manchas_solares.png}
\end{figure}

\item \textbf{¿Encuentras un solo ciclo principal o un conjunto de ciclos con frecuencia cercana? ¿Cuál sería el promedio del conjunto de frecuencias?}\\
Encontramos 4 ciclos que se encuentran juntos, estos ciclos tienen un promedio de número de manchas más alto de lo normal. Estos 4 ciclos poseen una frecuencia cercana, de alrededor de 11 años. Para el promedio de tiempo en años de los ciclos tenemos que es de 10.85605 años. \\

\item \textbf{¿Qué otros ciclos relevantes encuentras? Proporciona una tabla con las amplitudes de los ciclos.}\\
\begin{table}[h]
\centering
\label{my-label}
\begin{tabular}{|l|l|l|l|l|}
\hline
Ciclo & Periodo (Meses) & Periodo (Años) & Frecuencia       & Amplitud      \\ \hline
23    & 139.6956        & 11.6413        & 0.00715841892312 & 22.6781521056 \\ \hline
24    & 133.875         & 11.1562        & 0.00746965452848 & 39.9872332086 \\ \hline
25    & 128.52          & 10.71          & 0.00778089013383 & 35.4198320032 \\ \hline
27    & 119.0           & 9.9167         & 0.00840336134454 & 30.9080378809 \\ \hline
\end{tabular}
\end{table}

\item \textbf{Lo que han encontrado hasta ahora son ciertas regularidades, incluso hay pronósticos de un rango para el número de manchas solares. ¿Cómo crees que es posible predecir el número de manchas?}\\
En lo que han encontrado regularidades, es en el periodo de tiempo que tarde en llegar de un máximo (o un mínimo) a otro, le número de manchas solares tiende a variar también. Cada determinado número de tiempo hay otro gran ciclo n. Durante ese periodo de tiempo, el número de manchas solares comienza a aumentar a un ritmo dado y cuando este llega a su cúspide, el número de manchas solares comienza a descender a ese mismo ritmo como lo hace una onda, permitiendo que los observadores puedas realizar una predicción aproximada acerca del número de manchas que presentará el Sol en determinado tiempo.
\begin{figure}[h]
\includegraphics[scale=0.45]{Prediccion.png}
\end{figure}
\end{enumerate}


\end{document}
