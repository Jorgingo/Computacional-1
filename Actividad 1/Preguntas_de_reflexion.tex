\documentclass{article}
\usepackage[spanish]{babel}
\usepackage[utf8]{inputenc}
\usepackage{graphicx}
\usepackage{setspace}
\usepackage[usenames]{color}

\begin{document}
\title{Preguntas de reflexión}
\author{Campos Quiñonez Jorge Andres}
\maketitle

\begin{enumerate}
\item \textbf{¿Cual es tu primera impresión de uso de LaTeX?}\\
Si hablo de manera honesta, es un editor de texto que complica demasiado las cosas. Tiene que saber los comandos para poder realizar cualquier tipo de edición, por más sencillo que sea. Hasta para hacer un listado se utilizan 3 comandos (begin, item, end) y poder comenzar a desarrollar un texto completo, es bastante más
\item \textbf{¿Qué aspectos te gustaron más?}\\
Hasta ahora, casi nada. El conversor a PDF resultaría muy conveniente, ¿pero que editor de texto no lo incluye hoy en día?
\item \textbf{¿Qué no pudiste hacer en LaTeX?}\\
Hasta ahora, no sabría decir porque no le pedí casi nada, sólo lo más básico.
\item \textbf{En tu experiencia, comparado con otros editores, ¿cómo se compara LaTeX? }\\
Hasta ahora, me ha complicado las cosas, la única ventaja que le veo hasta ahora es el formato nativo que posee.
\item \textbf{¿Qué es lo que mas te llamó la atención en el desarrollo de esta actividad?}\\
Que me tomó más tiempo aprender a usar LaTex que desarrollar el tema requerido.
\item \textbf{¿Qué cambiarías en esta actividad?}\\
Es una buena actividad, pero agregaría otra en la cual se concentre en aprender a utilizar todo el potencial de LaTex, no aventarse de lleno pensando que ya se sabe todo.
\item \textbf{¿Que consideras que falta en esta actividad? }\\
Veasé respuesta anterior.
\item \textbf{¿Puedes compartir alguna referencia nueva que consideras util y no se haya contemplado? }\\
Google fue mi mejor amigo para encontrar comandos y como realizar ciertas cosas en LaTex.
\item \textbf{¿Algún comentario adicional que desees compartir? }\\
Me pasé más de 2 horas buscando que LaTex colocara el logo de la UNISON por encima del titulo en la portada, ya que le resulta muy complicado unir una imagen con texto.
\end{enumerate}





\end{document}